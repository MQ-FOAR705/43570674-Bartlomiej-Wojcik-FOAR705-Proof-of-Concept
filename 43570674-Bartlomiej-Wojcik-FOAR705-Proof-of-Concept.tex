\documentclass{article}
\usepackage[utf8]{inputenc}

\title{43570674-Bartlomiej-Wojcik-FOAR705-Proof-of-Concept}
\author{Bart Wojcik}
\date{12 September 2019}

\begin{document}

\maketitle

\section{User Stories}
\begin{enumerate}
    \item As a research student working with literature and fresh out of a bachelor's degree in Japanese studies, anxious about my still developing Japanese language skills, I want to be able to effectively engage with Japanese literature in original.
    \item As a research student working with Japanese text, I want to be able to collaborate with my supervisor on the same digital copy of a text.
    \item As a supervisor of a Japanese literature research project, I want to be able to check on my student's progress independent of our meeting schedule.
    \item As an undergraduate student interested in Japanese literature, I want to be able to start effectively reading Japanese texts sooner.
    \item As a teacher of Japanese language I want a tool I can give to my students to help them get comfortable with reading large texts in Japanese.
    \item As a researcher working with Japanese literature, I want to increase the speed at which I can read a large  text.
    \item As a reader of Japanese literature, I want spend less time looking words up in a dictionary and more time reading interesting texts.
    \item As a researcher working with Japanese literature, I want a portable research environment for working on digital texts. I want it to sit on my keyring USB pendrive and work out of the box regardless of what kind of a computer I plug it into.
    \item As members of a student group project working on a Japanese text, we want to be able to collaboratively annotate the same copy of a text as our peers easily. We want to be able to easily see each other's notes on the same text without having to take any steps. If one us leaves a note, we all want to see it.
    \item As a teacher of Japanese language, I want to see the notes my students make on a text I assign them, as they progress to it. I want to be able to monitor their comprehension and provide feedback as they go by monitoring the notes they take while reading.
\end{enumerate}

\end{document}
