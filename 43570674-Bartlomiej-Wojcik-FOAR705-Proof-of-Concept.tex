\documentclass{article}
\usepackage[utf8]{inputenc}

\title{43570674-Bartlomiej-Wojcik-FOAR705-Proof-of-Concept}
\author{Bart Wojcik}
\date{12 September 2019}

\begin{document}

\maketitle

\section{User Stories}
\begin{enumerate}
    \item As a research student working with literature and fresh out of a bachelor's degree in Japanese studies, anxious about my still developing Japanese language skills, I want to be able to effectively engage with Japanese literature in original.
    \item As a research student working with Japanese text, I want to be able to collaborate with my supervisor on the same digital copy of a text.
    \item As a supervisor of a Japanese literature research project, I want to be able to check on my student's progress independent of our meeting schedule.
    \item As an undergraduate student interested in Japanese literature, I want to be able to start effectively reading Japanese texts sooner.
    \item As a teacher of Japanese language I want a tool I can give to my students to help them get comfortable with reading large texts in Japanese.
    \item As a researcher working with Japanese literature, I want to increase the speed at which I can read a large  text.
    \item As a reader of Japanese literature, I want spend less time looking words up in a dictionary and more time reading interesting texts.
    \item As a researcher working with Japanese literature, I want a portable research environment for working on digital texts. I want it to sit on my keyring USB pendrive and work out of the box regardless of what kind of a computer I plug it into.
    \item As members of a student group project working on a Japanese text, we want to be able to collaboratively annotate the same copy of a text as our peers easily. We want to be able to easily see each other's notes on the same text without having to take any steps. If one us leaves a note, we all want to see it.
    \item As a teacher of Japanese language, I want to see the notes my students make on a text I assign them, as they progress to it. I want to be able to monitor their comprehension and provide feedback as they go by monitoring the notes they take while reading.
\end{enumerate}

\section{Acceptance Criteria}
NOTE: These correspond 1:1 to each of the user stories define in section 1.
\begin{enumerate}
    \item I want a program that will help me read difficult texts without using machine translation which I think is unreliable. I want to see the definitions and the pronunciation of words as I read them.
    \item I want a digital research environment that lets me and my supervisor operate on the same copy of one text simultaneously on two different machines. We want to be able to interact with the same copy of a text each on our own computer.
    \item I want a work environment that lets me see how my research student is interacting with the text she is working on at any time, regardless of whether my student is available or not. I want the program to show me what the student has been doing.
    \item My vocabulary is poor but I want to start reading in Japanese now. I want a program that will help me understand Japanese texts but is not a dictionary where I have to type in words I need look up.
    \item I want a software package I can easily hand out to students who will be able to just run it with minimum set-up. I want the software to do whatever it can to help intermediate level students read advanced texts sooner.
    \item I want a program that will help me increase my reading speed by reducing the number of times I have to reach for a dictionary.
    \item I want a program that makes looking up words in a dictionary less time consuming. I just want to see what the word means as soon as I put my cursor on it without having to right-click anything or perform iterative actions.
    \item I travel a lot and can't always take a computer with me but often have access to all kinds of weird computers, I want a research environment that sits on a USB drive and just works on all of them, or at least most of them.
    \item We just want a program that could emulate a scenario where we all leave notes on the margins of the same book, but on a computer.
    \item I want to be able to virtually hand out a copy of a text to my students in a program that lets me assign them tasks and monitor their progress as they go.
\end{enumerate}

\section{User Story Merge}
\begin{enumerate}
    \item I want a program that isn't a dictionary or a machine translator but helps me understand texts in Japanese.
    \item I want a environment for learning, teaching and research.
    \item I want a co-working platform that lets multiple users work with the same source material collaboratively.
    \item I want a ready-to-go work environment that has everything I and my students need. I want this to work out-of-the-box.
    \item I want a software package I can easily hand out to others.
    \item I want a portable work environment that keeps my stuff and works on whatever computer I plug it into, wherever I go.
\end{enumerate}

\section{Work Order Including Quality Assurance Measures}
\begin{enumerate}
    \item Build a blank environment.
    \item Add requested features - reading comprehension aids, annotation tools, collaboration tools.
    \item Pre-configure all included software and software extensions.
    \item Test functionality against each user scenario acceptance criteria on the primary development system.
    \item Add cross-platform binaries and configure each to use the same environment configuration and user files.
    \item Test on target platforms, determine whether the software can be executed and meets user acceptance criteria (in other words, make sure all the requested functionality is retained when the software is executed regardless of the platform it is executed on.
    \item Package up for distribution.
    \item Test deployment scenarios to ensure the final package successfully deploys on target systems, once again test against user acceptance criteria.
\end{enumerate}

\end{document}
